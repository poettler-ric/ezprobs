\section{Problems}

In this section the steps needed to create a new problem will be discussed.

\subsection{Skeleton}

In this subsection the bare minimum of steps will be discussed to create the
skeleton of a problem.

\subsubsection{Template}

The template controls how the problem is displayed on on the client. All
templates are basically \verb+HTML+ processed by the
\href{https://jinja.palletsprojects.com/}{Jinja} template engine.

To create a new problem a new file with a descriptive name must be created
under \verb+ezprobs/templates/problems+. This file should have the extension
\verb+.html+. The content of the file might look like something like this:

\begin{lstlisting}[language=html]


TEST


test description



test solution

\end{lstlisting}

The \verb+extends+ directive is mandatory and causes the generation of e.g. the
parameter section.  The \verb+title+, \verb+description+ and \verb+solution+
blocks are also mandatory. The \verb+title+ block is used to set the title of
the problem. The remaining blocks will be described in subsequent sections.

During this demonstration the file will be saved as \verb+ezprobs/templates/problems/test.html+.

\subsubsection{Python Code}

All problems are implemented as
\href{https://flask.palletsprojects.com/}{Flask} blueprints.

The problem python files should be saved under the \verb+ezprobs/problems+
directory with a representative name. The minimum code needed to implement a
problem is as follows:

\begin{lstlisting}[language=python]
from flask import Blueprint, render_template

bp = Blueprint("test", __name__)

@bp.route("/", methods=["POST", "GET"])
def index():
    return render_template("problems/test.html")
\end{lstlisting}

The name provided to the \verb+Blueprint+ constructer must be unique across the
whole application. The template provided to the \verb+render_template+
function must be the filename of the previously created template.

For this demonstration the file will be called \verb+ezprobs/problems/test.py+.

\subsubsection{Linking}

To make the problem available in the Flask application it must be linked in it.
This will be done by modifying \verb+ezprobs/__init__.py+ by adding the
following lines to make the demonstration example available:

\begin{lstlisting}[language=python]
import ezprobs.problems.test
app.register_blueprint(
    problems.test.bp, url_prefix="/problems/test"
)
\end{lstlisting}

This loads the newly created module and makes the blueprint available under the
given url. The problem should be acessible by pointing the browser to
\href{http://localhost:5000/problems/test/} if the development server is running.

The values must be changed to the names of the current problem to add.

By convention problems should be linked under \verb+/problems+.

To make the problem available in the menu bar a new entry must be added to the
the \verb+app.config["problems"]+ dictionary. To make The \verb+test+ problem
available a new \verb+Test Runs+ section is added and the problem is named
\verb+Test+.

\begin{lstlisting}[language=python]
app.config["problems"] = {
    "Hydraulics": {
        "Flow Regime Transition": "flow_regime_transition_fit_3",
        "Pressure Pipe": "pressure_pipe",
    },
    "Mathematics": {
        "XY Problem": "xy",
    },
    "Test Runs": {
        "Test": "test",
    },
}
\end{lstlisting}

\subsection{Description}

The \verb+description+ block in the template is used to describe the problem at hand.

\subsection{Parameters}

\subsection{Solution}

\subsection{Images}

\subsubsection{Static Images}

\subsubsection{Graphs}

\subsubsection{Vector Graphics}

\subsubsection{Plots}

\subsection{Mathematical Expressions}

To render mathematical expressions \href{https://www.mathjax.org/}{MathJax} is
used. This enables the usage of \LaTeX\ in the \verb+description+ and
\verb+solution+ blocks.

To use the \LaTeX\ math mode for inline expressions they have to be enclosed in \verb+\(\)+ like:

\begin{lstlisting}
The discharge is given by \(Q\).
\end{lstlisting}

To render a single line equation it is enough to enclose it with \verb+$$+ like:

\begin{lstlisting}
$$f(x) = a \cdot x + b$$
\end{lstlisting}

For multiline equations an align \LaTeX\ environment should be used like this:

\begin{lstlisting}
$$
\begin{align}
f(x) &= a \cdot x + b \\
g(x) &= c \cdot x + d \\
\end{align}
$$
\end{lstlisting}
